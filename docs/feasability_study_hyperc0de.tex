\documentclass[a4paper]{article}

%NOTE: if XeLaTeX or LuLaTeX is installed you may uncomment the first two comments
%related to fonts in order to typeset this document in a more modern font

\usepackage{mathtools}
\usepackage[explicit]{titlesec}
\usepackage{titling}
\usepackage{vaucanson-g}
%\usepackage{fontspec}
\usepackage{fancyhdr}
\usepackage{multicol}
\usepackage{amssymb}
\usepackage{xcolor}
\usepackage{hyperref}
\usepackage{listings}
\lstset{basicstyle=\ttfamily,
  showstringspaces=false,
  commentstyle=\color{gray},
  keywordstyle=\color{blue}
}

%\setmainfont{Helvetica Neue Light}

\newcommand{\coursename}{WebAtelier}
\newcommand{\assingmentnr}{\underline{HYPER}\texttt{C0DE} }
\newcommand{\realauthor}{Tommaso dal Sasso}
\author{Client: \realauthor \\
Universit\`a della Svizzera italiana - \coursename}


\pagestyle{fancy}
\fancyhf{}
\rhead{\realauthor}
\lhead{\today}
\chead{\assingmentnr  - \coursename}
\cfoot{\thepage}
\titleformat{\section}{\normalfont\Large}{}{0em}{#1\ }
\titleformat{\subsection}{\normalfont\large}{}{0em}{#1\ }
\title{\assingmentnr}

\pretitle{\begin{flushright}\LARGE}
\posttitle{\par\end{flushright}\vskip 0.5em}
\preauthor{\begin{flushright}\large \lineskip 0.5em}
\postauthor{\par\end{flushright}}
\predate{\begin{flushright}\large}
\postdate{\par\end{flushright}}
\begin{document}

\maketitle{\raggedleft}
\emph{A Project by:}\\
Alexander Camenzind, Robert Talling, Valerio Battaglia and Simon Reding\\

Repository: \href{https://github.com/sreding/hypercode/}{github.com/sreding/hypercode/}

\section{Introduction}
Hypercode is a 3D browser based Integrated Development Unit designed specifically to display relationships between individual files of code. It is composed of three "states", the overview, the context view and the zoomed context view.
\subsection{Overview}
The Overview displays all files of a project in relationship to each other in a graph form. Graph nodes are clickable and link to the corresponding context view.
\subsection{Context View}
The context view allows a user to toggle between all related files of one selected main file. On the vertical axis there are all the files that are dependent on the main file. On the other hand, the horizontal axis contains all the files the the main file controls. A user can bring files from either axis into the foreground by using the arrow button. To set a file as the main file, a user simply clicks the focus button on a file. 

\subsection{Zoomed Context view}
This allows the user to inspect a file close by "zooming" into the file while keeping all the functionalities of the Context View.

\subsection{Editor}
The editor allows basic editing with live syntax highlighting as well as save and delete functionality. The editor can be reached from the context view and returns to the context view. Upon deletion of a file the overview is shown and all dependencies to/from that file are also deleted.

\section{Running the server}
As we are using a webpacker running the server takes some effort, and to run the server in development mode one requires a CORS plugin for the browser. To install the dependencies, npm has to be installed and
\begin{lstlisting}[language=bash]
npm install
\end{lstlisting}
has to be run in both \texttt{/frontend} and the project root.

\subsection{Deployable mode}
\begin{enumerate}
\item Compile the code in \texttt{/frontend}
\begin{lstlisting}[language=bash]
npm run build
\end{lstlisting}
\item Copy the files: (assuming you are in the project root)
\begin{lstlisting}[language=bash]
cp frontend/index.html public/index.html
#make public/dist/ if inexistent:
mkdir public/dist
cp frontend/dist/* public/dist/
\end{lstlisting} 
\item Run the server in the project root:
\begin{lstlisting}[language=bash]
npm start
\end{lstlisting}
\end{enumerate}
You should now be able to see the site at \texttt{http://localhost:3000/}
\subsection{Development mode}
\begin{enumerate}
\item Run the server as in the deployable mode (without compiling/copying the files)
\item Run the front end server in \texttt{/frontend}:
\begin{lstlisting}[language=bash]
npm run dev
\end{lstlisting}
\end{enumerate}
You should now be able to see the site at \texttt{http://localhost:8080/} with hot reload whenever the code is changed and saved. 
%\begin{multicols}{2}
\section{Libraries}
Over the past few weeks we have played around with a number of libraries to see what works and what doesn't. Here is our opinion on how these fit into our project.
\subsection{\href{https://get.webgl.org/}{WebGL} vs \href{https://developer.mozilla.org/en-US/docs/Web/CSS}{CSS}}
While the idea of WebGL may seem appealing, CSS is much more compatible with browsers and not as heavy. In addition, CSS allows us to use standard html components that can easily be seamlessly integrated into a website.
\subsection{\href{https://vuejs.org/}{WebGL}  vs \href{https://www.polymer-project.org/1.0/}{Polymer}}
We have decided to go with Vue.js as we have already learned about polymer in class and thought it would be a challenge to try something new. We believe Polymer could be used just as well if not better for our components.
\subsection{\href{http://minimal.be/lab/Sprite3D/}{Sprite3D.js}}
We started off using Sprite3D.js for our CSS3D transforms. But as we wanted to have more complex rotations, Sprite3D didn't really provide any support for that (since it is basically just a wrapper for CSS transforms.
Due to this reason we   switched to three.js.
\subsection{\href{https://codemirror.net/}{CodeMirror}}
CodeMirror allows basic editing. However, since it is an extremely custom component, we could not get the editing to properly work in the zoomed view.
\subsection{\href{https://highlightjs.org/}{Highlight.js}}
In contrast to CodeMirror, Highlight.js allows syntax highlighting in a standard HTML div but it has to be refreshed every time the code is changed, hence the cursor moves to the start of the text which in unsuitable for editing.
\subsection{\href{https://threejs.org/}{three.js}}
Three.js and its CSS3DRenderer provided all the 3D functionalities needed and we used Tween.js to animate the transormations. 
\\One problem we encountered between Vue and Three.js, was that we needed to create elements using a Vue v-for but then to use the elements with Three.js we had to remove them from the Vue template, which caused some problems. \\
Even though  we managed to solve this problem, It propably would have been a better idea to not make the context view a Vue component.
\subsection{\href{https://d3js.org/}{D3.js}}
D3.js is the library we have used to draw the force graph of the overview.
\section{Known Issues}
\enumerate
\item Reload fails with outside of development build due to a routing error.
\enumerate

%\end{multicols}
\end{document}